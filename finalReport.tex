\documentclass[aps, reprint, amsmath, amssymb]{revtex4-1}

\usepackage{amsmath}
\usepackage{hhline}
\usepackage{graphicx}
\graphicspath{ {images/} }

\usepackage[artemisia]{textgreek}


\begin{document}
\title{A Comparison of Machine Learning Techniques for Classification of the Higgs Boson}
\author{Jared Moskowitz \& Julia Rowe}
\date{8 May 2015}

\maketitle

\section{Introduction}
One of the integral tasks of particle physics experiments is identifying elementary particles produced in high energy collisions. This can be especially problematic for the Higgs boson, given that the primary observed decay mode, W-W+, results in the production of quark jets, which are difficult to distinguish from the background. Machine learning techniques are used to distinguish the signal from the background.  The techniques depend on the features as defined by the standard model of the particles in question.  The spped and the accuracy of the algorithm is of utmost importance as the data can be very large.  The purpose of this analysis is to compare machine learning algorithms, looking at both the accuracy and how they scale with the number of features.  

\section{Methods}
Various algorithms from the Waikato Environment for Knowledge Analysis (WEKA) suite of machine learning software were tested under the same conditions.  The machines were trained using Monte Carlo data and tested on real data from ATLAS.  The Monte Carlo data was sorted into runs with weights creating a sample that matches that of real data.  Data was separated into four groups: even runs with nine features, odd runs with nine features, even runs with thirty-one features, and odd runs with thirty-one features.  Predictive machine models were built from each of these groups and were used to test the opposite group of runs with the same number of features.  

\subsection{The Algorithms}
\subsubsection{Decision Trees}
Decision trees creates a flowchart-like structure consisting of three types of nodes. The most important of these nodes are the decision nodes which act as tests or comparisons of a feature of an event.  The branches represent the outcome of the test, which continues down until an end node or leaf is reached, representing a classification.  

\subsubsection{Random Forest}
Random forests are made up of multiple decision trees using random feature selection and bootstrap aggregation to correct for the common problem of decision tree overfitting.  This algorithm creates multiple decision trees by selecting a smaller subset of the features and building trees. Our random forests created ten trees.  For runs with nine features each tree selected four of these features and for runs with thirty-one features six features were selected.

\subsubsection{Na{\"i}ve Bayes}
This algorithm creates simple probabilistic classifiers by applying Bayes' theorem under the assumption of statistical independence between the features.  Although highly simplified, the Na{\"i"}ve Bayes algorithm has been successful in highly complex situations.  It is also highly scalable, making it an efficient algorithm.  

\subsubsection{Support Vector Machine}
Support vector machines builds a hyperplane that best separate the two categories of data. This hyperplane is built from support vectors from the training data.  An advantage of SVM is it uses a fixed number of support vectors from the sample, avoiding overfitting.  Often the data is projected into higher dimensional feature space, if the set is not separable in the given dimension.  The hyperplane is defined the set of points \textbf{x} which satisfy the equation:

\begin{equation}
\mathbf{w} \cdot \mathbf{x} - b = 0
\end{equation}

where \textbf{w} is the vector normal to the hyperplane.  Anything above the hyperplane should have the label of 1 and anything below -1.  

\section{Results}

\begin{center}
  \begin{tabular}{ | l || l | l | l | l |}
    \hline
    Algorithm & 9 Even & 9 Odd & 31 Even & 31 Odd \\ \hhline{|=||=|=|=|=|}
    Decision Tree & 10.056s & 8.605s & 26.672s & 29.490s \\ \hline
    Random Forest & 21.346s & 23.487s & 37.249s & 38.542s \\ \hline
    Na{\"i}ve Bayes & 1.381s & 1.687s & 2.989s & 3.099s \\ \hline
    SVM - linear & 17.110s & 17.572s & 86.40s & 82.086s \\ \hline
    \hline
  \end{tabular}
\end{center}


\begin{center}
  \begin{tabular}{ | l || l | l | l | l |}
    \hline
    Algorithm & 9 Even & 9 Odd & 31 Even & 31 Odd \\ \hhline{|=||=|=|=|=|}

    Decision Tree & 88.41 \% & 87.98\% & 87.70\% & 87.75\% \\ \hline

    Random Forest & 88.72\% & 88.57\% &  88.82\% & 88.53\% \\ \hline

    Na{\"i}ve Bayes & 87.52\% & 87.79\% & 84.94\% & 84.97\% \\ \hline

    SVM - linear & 87.07\% & 87.01\% & 87.33\% & 87.26\% \\ \hline
    \hline
  \end{tabular}
\end{center}

\begin{center}
  \begin{tabular}{ | l || l | l | l | l |}
    \hline
    Algorithm & 9 Even & 9 Odd & 31 Even & 31 Odd \\ \hhline{|=||=|=|=|=|}

    Decision Tree & 0.907 & 0.904 & 0.855 & 0.861 \\ \hline

    Random Forest & 0.929 & 0.928  & 0.929  & 0.929 \\ \hline

    Na{\"i}ve Bayes & 0.931  & 0.929 & 0.915 & 0.913 \\ \hline

    SVM - linear & 0.792 & 0.794 & 0.803 & 0.803 \\ \hline
    \hline
  \end{tabular}
\end{center}


%\begin{figure}[t]
%    \centering
%    \includegraphics[width=.8\columnwidth]{IMAGE FILE NAME GOES HERE}
%    \captionof{figure}{
%    CAPTION GOES HERE}
%\end{figure}

\section{Conclusion}



\end{document}
